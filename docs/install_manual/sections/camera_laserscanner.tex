
\section{Additional (real) Hardware: Camera and Laserscanner}

\subsection{Step by step to install: Camera, LpzRobots and Eclipse, Laser scanner}
1) Linux: Install \emph{Linux} e.g., Ubuntu 12.04\\
2) Camera: Install \emph{OpenCV} and \emph{AprilTag} for Camera (see How to Use Camera.pdf) OR follow the steps below
\begin{enumerate}
\item Use Synaptic Package Manager to install (OpenCV): libopencv-dev
\item Install necessary packages of AprilTag:

\begin{lstlisting}

See http://april.eecs.umich.edu/wiki/index.php/Download_and_Installation

OR

sudo apt-get install emacs git-core ant subversion gtk-doc-tools
libglib2.0-dev libusb-1.0-0-dev gv libncurses-dev openjdk-6-jdk
autopoint libgl1-mesa-dev

\end{lstlisting}

\item Install LCM of AprilTag:

\begin{lstlisting}
cd $HOME
svn checkout http://lcm.googlecode.com/svn/trunk lcm
cd lcm
./bootstrap.sh
./configure
make
sudo make install
\end{lstlisting}

\item Install libdc1394of AprilTag:

\begin{lstlisting}
cd $HOME
svn co https://libdc1394.svn.sourceforge.net/svnroot/libdc1394/trunk/libdc1394/
cd libdc1394
autoreconf -i -s
./configure
make
sudo make install

OR

Note that I met with a problem when I install libdc1394.
So, another way to do this is:
At terminal: aptitude search libdc1394
then you can see several files there.
Choose libdc1394-22 and libdc1394-22-dev to install, like:
sudo aptitude install libdc1394-22
sudo aptitude install libdc1394-22-dev
\end{lstlisting}

\item Set up environment variables of AprilTag by adding the following lines to .bashrc
\begin{lstlisting}
export CLASSPATH=$CLASSPATH:/usr/share/java/gluegen-rt.jar:/usr/
local/share/java/lcm.jar:$HOME/april/java/april.jar:./
export LD_LIBRARY_PATH=$LD_LIBRARY_PATH:$HOME/april/lib
alias java='java -ea -server'
\end{lstlisting}

\item Then reload your bash settings (you won't need to do this the next time you log in):
\begin{lstlisting}
source ~/.bashrc
\end{lstlisting}

\item Install April Toolkit:
\begin{lstlisting}
cd $HOME
git clone git://april.eecs.umich.edu/home/git/april.git
cd april/java
ant
\end{lstlisting}

\item Test AprilTag $->$ You can test the 3D visualization environment with:
\begin{lstlisting}
java april.vis.VisTest

Or try out the camera acquisition software with:

java april.jcam.JCamView

Or try to run camera test program at:
\camera\ToSteffen
Make sure that "cvCaptureFromCAM(-1) or cvCaptureFromCAM(0)"
in test_client.cpp

Terminal 1:
java j_server_test

Terminal 2:
make clean
make
./test_client
\end{lstlisting}

\end{enumerate}

3) \emph{Lpzrobots and Eclipse} (see Setting up Eclipse and LpzRobots section)
\begin{lstlisting}
If you want to use Laser scanner then make sure that in
./setUpGoRobots.sh

These ones are not installed:
#echo "sudo apt-get install binutils-gold"
#sudo apt-get install binutils-gold

>After finish this:
>Start: Eclipses>>Select: Help>Install Software

1. Work with: http://download.eclipse.org/tools/cdt/releases/indigo
Select: C/C++ Development Tools

2. Work with: http://download.eclipse.org/egit/updates
Select: Eclipse EGIT


Due to Ubuntu 12.04, you need to type sudo make all
then you might get the error messages:

make[2]: Leaving directory `/home/poramate/workspace/pmanoonpong-lpzrobots-fork'
sudo make install_ode_robots
[sudo] password for poramate:


> ./start
> ./start: error while loading shared libraries: libode\_dbl.so.1:
  cannot open shared object file: No such file or directory

The solution was:
When installing Lpzrobots make sure that you install it in
/home/yourlogin, e.g., /home/poramate

If you made this mistake, you could change this by:
From terminal, go to your lpzrobots,
e.g. poramate@ubuntu:~/workspace/pmanoonpong-lpzrobots-fork$

make clean
make clean-all
make conf

Then: set new installing path: /home/yourlogin, e.g., /home/poramate

make all

\end{lstlisting}

4) \emph{Laser scanner}

\begin{lstlisting}
utilities/urg
1) sudo apt-get install libsdl-net1.2
2) sudo apt-get install libsdl-net1.2-dev
3) sudo apt-get install libsdl1.2-dev

4) sudo sh configure
5) sudo make
6) sudo make install

poramate@ubuntu:~/Documents/utilities/urg


Go to example:

poramate@ubuntu:~/Documents/utilities/urg/samples/cpp


Running program:

UrgDevice::connect: /dev/ttyACM0: No such file or directory
poramate@ubuntu:~/Documents/utilities/urg/samples/cpp: sudo ./mdScan
UrgDevice::connect: /dev/ttyACM0: No such file or directory
poramate@ubuntu:~/Documents/utilities/urg/samples/cpp:

\end{lstlisting}

\subsection{Starting Laser scanner demo with simulated and real robots}
\begin{lstlisting}
1) Check out "preview_laserscanner"  branch from Lpzrobots repository
2) in the Lpzrobot directory: e.g., ~/worskspace/ yourname-lpzrobots-forks: make conf , 
select option = 1 for using laserscanner, 
select option = 0 for not using laserscanner,
3) then make all
4) go to ~/worskspace/ yourname-lpzrobots-forks/oderobots/
simulations/template_laserscanner
5) type make and then ./start
\end{lstlisting}