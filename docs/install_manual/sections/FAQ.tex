\section{FAQs}

\subsection{Installation errors in general}

\subsubsection{Symbol lookup error}
After installing the "binutils-gold" package, I get the following error after recompiling my source code:
"./test: symbol lookup error: /usr/local/lib/liburg.so.0: undefined symbol: \_ZTIN3qrk10CoordinateE"
(Note: liburg is a library for the Hokuyo laser scanners from here or via packet manager).\\

As "binutils-gold" contains a different linker for C++ files, which is optimized for speed, it is likely that this new linker does not link the laser scanner libraries correctly (probably due to "bad" programming within the drivers).
However, this new linker is not necessary for compiling and running lpzrobots, it just speeds up the compilation time.
So the solution to the problem is simple: Uninstall "binutils-gold" and everything works.


\subsection{Errors using setUpGoRotobots.sh}

\subsubsection{Error with \emph{git clone}}

Whilst cloning the repositories, I encountered the following error:
\begin{lstlisting}
git clone https://crauterb@git.assembla.com/lpzrobots.git -b master
Cloning into crauterb-lpzrobots-fork...
Password:
remote: Counting objects: 25478, done.
remote: Compressing objects: 100% (6030/6030), done.
remote: Total 25478 (delta 19211), reused 25478 (delta 19211)
Receiving objects: 100% (25478/25478), 19.97 MiB | 527 KiB/s, done.
Resolving deltas: 100% (19211/19211), done.
warning: Remote branch master not found in upstream origin, using HEAD instead
warning: remote HEAD refers to nonexistent ref, unable to checkout.
\end{lstlisting}

the repository Lpzrobots when there was no branch \emph{master}, and for some reason, it was not added later and did not appear anywear.
The solution was simple: Delete the forked version and create a new one - if possible. Worked for me.

\subsection{Problems with GIT}

\subsubsection{Setting up Repositories in Eclipse}
\label{EclipseGIT}
In some cases, the instructions on how to set up the GIT-repositories within Eclipse did not work. \\
Here is a different approach, that worked for me: \\
\begin{enumerate}
 \item Import the repositories into the GIT-view of Eclipse, just as described before
 \item Instead of importing over the GIT-View, you now go onto \emph{File $\rightarrow$ Import $\rightarrow$ General $\rightarrow$ Existing Projects into Workspace} and you then choose the two repositories
 \item After Eclipse has imported the files, you can \emph{right-click} on the Project, and then select \emph{Team $\rightarrow$ Share}
 \item Now, just select \emph{GIT} and the two GIT-repository-adresses should appear
 \item \emph{Apply}
\end{enumerate}


\subsubsection{No Connection to GIT Server: The remote end hung up unexpectedly.}

If you defined stable as usual (check with{\tt git remote -v}):\\
 \nolinkurl{stable}    \nolinkurl{https://wbj@git.assembla.com/lpzrobots.git (fetch)}\\
 \nolinkurl{stable}    \nolinkurl{https://wbj@git.assembla.com/lpzrobots.git (push)}\\\\
%
but when typing {\tt git fetch origin} you get the following error message: \\
{\tt Password for 'https://wbj@git.assembla.com':} \\
{\tt error: RPC failed; result=22, HTTP code = 401} \\
{\tt fatal: The remote end hung up unexpectedly} \\\\
%
and also defining the origin with http only:\\
\nolinkurl{stable_http}    \nolinkurl{http://wbj@git.assembla.com/lpzrobots.git (fetch)}\\
does not help, try to use the git protocol:\\
\nolinkurl{stable_git2}    \nolinkurl{git@git.assembla.com:lpzrobots.git (fetch)}\\
\nolinkurl{stable_git2}      \nolinkurl{git@git.assembla.com:lpzrobots.git (push)}

\subsection{Compile Errors}

\subsubsection{OpenCV highgui missing}
You are using gorobots as a library build (``new makefile system'') and you
are experiencing the following error while trying to compile, you are missing
the highgui library development files of the opencv package:
\begin{lstlisting}
utils/real_robots/sensors/camera/CamTagPositionSensor.h:11:39: 
fatal error: opencv2/highgui/highgui.hpp: No such file or directory
\end{lstlisting}
Install the missing package by typing:
\begin{lstlisting}
$ sudo apt-get install libopencv-highgui-dev
\end{lstlisting}

\subsubsection{C++11 unknown}
If you are trying to compile and obtain errors related to the flag 
\begin{lstlisting}
 -std=c++11
\end{lstlisting}
it means that you finally will have to upgrade your machine as your version of
gcc does not support the C++11 flag.

\subsubsection{osg::Geometry::BIND\_PER\_PRIMITIVE unknown}
If your version of Open Scene Graph it too recent, you may observe the following
error while trying to compile lpzrobots:
\begin{lstlisting}
osg/osgprimitive.cpp:668:32: error: 
  'BIND_PER_PRIMITIVE' is not a member of 'osg::Geometry'
\end{lstlisting}
One quick fix for this is to just comment out line 668 in osg\/osgprimitive.cpp.
Updating to a more recent version of lpzrobots would resolve the problem as 
well.


\subsubsection{cannot find -ltinyxml2}
With the recent version of gorobots where unit test is embedded, you might have this problem after ``make'' in gorobots:
\begin{lstlisting}
/usr/bin/ld: cannot find -ltinyxml2
collect2: error: ld returned 1 exit status
make: *** [libgorobots.so] Error 1
\end{lstlisting}
You need to install -ltinyxml2. For installation, you could use synaptics and search for tinyxml2 and then select "libtinyxml2-dev and litinyxml2-0.0.0".


\subsection{Errors when starting the simulation}

The error messages was:

\begin{lstlisting}
> ./start
> ./start: error while loading shared libraries: libode\_dbl.so.1:
  cannot open shared object file: No such file or directory
\end{lstlisting}
The solution was:
\begin{lstlisting}
 > source ~/.bashrc
\end{lstlisting}


\begin{lstlisting}
if you use Ubuntu version 12.4 LTS, you might see this problem:

> ./start
> ./start: symbol lookup error: /usr/lib/libgsl.so.0: undefined symbol: cblas_dnrm2
\end{lstlisting}
The solution was:
\begin{lstlisting}
 > sudo apt-get remove binutils-gold
\end{lstlisting}



\subsection{Using Lpzrobots}
\begin{lstlisting}
> after start simulation , press 1 to fixed camera view
> Ctrl r = record movie
> .\start -f = record log file
> .\start = start program
> .\start -g 1 = display GUI
\end{lstlisting} 

\subsection{Using EPuck Monitor}

\subsubsection{When trying to run qmake, I obtain ``unknown QT: widgets''}
EPuck monitor is optimized for Qt5. If you obtain the ``unknown QT: widgets'' warning you are most likely using Qt4 by default on your system. Find out how to enable Qt5 by default on your system. If you are working with Ubuntu 12.04 LTS, the following terminal commands are what you are looking for. They will install Qt5 and enable it as default:
\begin{lstlisting}
$ sudo apt-add-repository ppa:ubuntu-sdk-team/ppa
$ sudo apt-get update
$ sudo apt-get install qtdeclarative5-dev qt5-default
\end{lstlisting}


\subsection{Serial interface}
\begin{lstlisting}
> sudo usermod -a -G dialout username
> Then restart your PC

\end{lstlisting} 
