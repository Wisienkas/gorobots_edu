\section{Using the Google Test Framework for Unit Testing}

We are trying to maintain our code with the help of a unit testing framework, 
namely googletest. 

\subsection{Installing the framework}

To be able to use this feature and develop new test cases, you first have to 
install the googletest framework. 
Here are the necessary steps:

Start by downloading the source code of the most recent googletest version. 
It is available under \url{https://code.google.com/p/googletest/downloads}. 
Decompress the archive and enter it in a terminal:
\begin{lstlisting}
$ cd gtest-1.7.0
\end{lstlisting}
Inside the decompressed folder, create a new directory called ``build'' and enter it:
\begin{lstlisting}
$ mkdir build
$ cd build
\end{lstlisting}
Use cmake to create a Makefile and use this Makefile to build googletest:
\begin{lstlisting}
$ cmake ..
$ make
\end{lstlisting}
You'll find a libgtest.a and libgtest\_main.a file inside the build directory. 
Copy these files to some path that is registered in the LD\_LIBRARY\_PATH 
environment variable. Typically, during the installation of lpzrobots, you add 
the lib folder in your home directory to the LD\_LIBRARY\_PATH. 
To copy the files there, type:
\begin{lstlisting}
$ cp libgtest.a libgtest_main.a ~/lib
\end{lstlisting}
Furthermore, you have to copy the folder gtest inside the include directory 
located in the googletest root folder to some path that you include into the 
search path while compiling gorobots simulations. 
You can use the include folder created during the lpzrobots installation 
(typically in your home folder). Starting in the build directory, type the 
following:
\begin{lstlisting}
$ cd ../include
$ cp -r gtest ~/include/
\end{lstlisting}
This finishes the installation of the googletest framework. 
You may now enter the gorobots folder and build the test cases:
\begin{lstlisting}
$ cd ~/workspace/gorobots
$ make test
\end{lstlisting}
You should also try to run the tests:
\begin{lstlisting}
$ ./build/run_test
\end{lstlisting}

\subsection{Creating test cases}
There is a nice tutorial about the basic ideas and concepts of the googletest
framework at the webpage of the project. 
You should definitely read it before going on:
\url{https://code.google.com/p/googletest/wiki/V1_7_Primer}

Within the gorobots project, we want to use the googletest framework for real 
unit testing as well as for something that is more like integration testing.
Unit testing refers to the testing of as small as possible subunits of code.
For instance, there is a test for the ann-framework that checks whether the 
activity property of the Neuron class works as expected:
\begin{lstlisting}
  TEST(NeuronTest, activity) {
    Neuron neuron;
    neuron.setActivity(1.2);
    ASSERT_EQ(1.2, neuron.getActivity());
  }
\end{lstlisting}
This is as primitive as it can get. Observe that ``NeuronTest'' is the name of 
the test case and ``activity'' the name of the actual test. You should make sure
your test case names are unique within gorobots.

For creating a test for a given unit of code, create a new file in the  
``tests''  folder within the gorobots repository.
The path of the file within the tests folder should resemble the path of the 
tested unit in the gorobots repository.
Furthermore, the name of the test file should be the name of the unit plus 
``\_test''.
If, for instance, the files containing the actual code are
\begin{lstlisting}
  utils/ann_framework/neuron.h
  utils/ann_framework/neuron.cpp
\end{lstlisting}
the correct path and name of the corresponding test file is
\begin{lstlisting}
  tests/utils/ann_framework/neuron_test.cpp
\end{lstlisting}
Within your new test file, you have to include the header of the googletest
framework and the headers necessary to perform your tests. For instance:
\begin{lstlisting}
  #include "gtest/gtest.h"
  #include "utils/ann-framework/neuron.h"
\end{lstlisting}
Afterwards, just start to define your tests.

When it comes to testing of projects, thins get a little more trickier.
This is due to our standard of defining the simulation class within the 
main.cpp.
Unfortunately, every main.cpp typically also contains one main method and we 
cant't link multiple objects that contain methods with the same name.
The solution for this is a little hackish:
Before including the main.cpp, which is necessary here, add two define 
statements to override the name of the main method and the ThisSim class.
The latter is neccessary to avoid identically named classes in different object
files.
The results might look like this:
\begin{lstlisting}
  #include "gtest/gtest.h"

  // mask names of objects in global scope
  #define main projects_nejihebi_example_main
  #define ThisSim ProjectNejihebiExampleMainSim

  #include "projects/nejihebi/example/sim/main.cpp"
\end{lstlisting}
Within the actual test of your project, you should run your simulation for an as
short as possible time interval and check afterwards whether the relevant 
parameters of the robot have the expected values.
This might require too add some additional methods to the ThisSim class.
Check out the test of the Nejihebi example project to get an idea how this might
look like:
\begin{lstlisting}
  tests/projects/nejihebi_example_test.cpp
\end{lstlisting}
